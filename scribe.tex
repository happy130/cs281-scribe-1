\documentclass[submit]{harvardml}

% Put in your full name and email address.
\name{Tian Lu}
\email{tianlu@g.harvard.edu}


% You don't need to change these.
\usepackage{url, enumitem}
\usepackage{amsfonts}
\usepackage{listings}
\usepackage{bm}
\usepackage[pdftex]{graphicx}

% Some useful macros.
\newcommand{\given}{\,|\,}
\newcommand{\R}{\mathbb{R}}
\newcommand{\E}{\mathbb{E}}
\newcommand{\var}{\text{var}}
\newcommand{\cov}{\text{cov}}
\newcommand{\N}{\mathcal{N}}
\newcommand{\ep}{\varepsilon}

\newcommand{\Dir}{\text{Dirichlet}}

\begin{document}

%%%%%%%%%%%%%%%%%%%%%%%%%%%%%%%%%%%%%%%%%%%%%%%%%%%%%%%%
%%%%%%%%%%%%%%%%% PROBLEM 1 %%%%%%%%%%%%%%%%%%%%%%%%%%%%
%%%%%%%%%%%%%%%%%%%%%%%%%%%%%%%%%%%%%%%%%%%%%%%%%%%%%%%%
\begin{problem}[Naive Bayes problem sets]
\begin{enumerate}[label=(\alph*)]In this problem, we look at maximum likelihood parameter estimation using the naive
Bayes assumption. Here, the input features $x_j, j = 1, . . . , n$ to our model are discrete,
binary-valued variables, so $x_j \in \{0, 1\}$. We call $x = [x_1,  x_2 , \dots  x_n]^T$ to be the input vector.
For each training example, our output targets are a single binary-value $y ∈ {0, 1}$. Our
model is then parameterized by $\theta_{j|y=0} = p(x_j = 1|y = 0), \theta_{j|y=1} = p(x_j = 1|y = 1)$, and
$\theta_y = p(y = 1)$. We model the joint distribution of $(x, y)$ according to
\begin{align*}
P(y)&=(\theta_y)^y (1-\theta_y)^{1-y}\\
p(x|y=0)&=\prod_{j=1}^n p(x_j|y=0)\\
p(x|y=1)&=\prod_{j=1}^n p(x_j|y=1)\\
\end{align*}
\item  Find the joint likelihood function $ L=log \prod_{i=1}^m p(x^{(i)},y^{(i)};\theta)$ in terms of the model parameters given above. Here, $\theta$ represents the entire set of parameters$\{\theta_y,\theta_{j|y=0},\theta_{j|y=1},j=1,\dots,n\}$.
\item Derive the parameters $\theta_y,\theta_{j|y=0},\theta_{j|y=1}$ which maximize the likelihood function 
\end{enumerate}
\vspace{0.1cm}
\end{problem}
\begin{enumerate}[label=(\alph*)]
\item 
\begin{align*}
L&=log \prod_{i=1}^m p(x^{(i)},y^{(i)};\theta)\\
&=log \prod_{i=1}^m p(x^{(i)}|y^{(i)};\theta)P(y^{(i)};\theta)\\
&=log \prod_{i=1}^m(\prod_{j=1}^n p(x_j^{(i)}|y^{(i)};\theta))P(y^{(i)};\theta)\\
&=\sum_{i=1}^m (\sum_{j=1}^n log p(x_j^{(i)}|y^{(i)};\theta)+log P(y^{(i)};\theta)\\
&=\sum_{i=1}^m (\sum_{j=1}^n(x_j^{(i)}log\theta_{j|y^{(i)}}+(1-x_j^{(i)})log(1-\theta_{j|y^{(i)}})+y^{(i)}log\theta_y+(1-y^{(i)}log(1-\theta_y))\\
\end{align*}

\item\begin{align*}
\nabla_{\theta_{j|y=0}}L&=\nabla_{\theta_{j|y=0}}\sum_{j=1}^m(x_j^{(i)}log\theta_{j|y^{(i)}}+(1-x_j^{(i)})log(1-\theta_{j|y^{(i)}})\\
&=\nabla_{\theta_{j|y=0}}\sum_{j=1}^m(x_j^{(i)}log\theta_{j|y=0}1\{y^{(i)}=0\}+(1-x_j^{(i)})log(1-\theta_{j|y=0})1\{y^{(i)}=0\})\\
&=\sum_{j=1}^m (\frac{x_j^{(i)}1\{y^{(i)}=0\}}{\theta_{j|y=0}}+\frac{(1-x_j^{(i)})1\{y^{(i)}=0\}}{1-\theta_{j|y=0}})\\
0&=\sum_{j=1}^m (\frac{x_j^{(i)}1\{y^{(i)}=0\}}{\theta_{j|y=0}}+\frac{(1-x_j^{(i)})1\{y^{(i)}=0\}}{1-\theta_{j|y=0}})\\
&=\sum_{j=1}^m (x_j^{(i)}(1-\theta_{j|y=0})1\{y^{(i)}=0\}+(1-x_j^{(i)})\theta_{j|y=0}1\{y^{(i)}=0\})\\
&=\sum_{j=1}^m((x_j^{(i)}-\theta_{j|y=0})1\{y^{(i)}=0\})\\
&=\sum_{j=1}^m (x_j^{(i)}1\{y^{(i)}=0\})-\sum_{j=1}^m (\theta_{j|y=0})1\{y^{(i)}=0\}\\
&=\sum_{j=1}^m (1\{x_j^{(i)}=1\wedge y^{(i)}=0\})-\theta_{j|y=0} \sum_{j=1}^m 1\{y^{(i)}=0\}\\
\theta_{j|y=0} &=\frac{\sum_{j=1}^m (1\{x_j^{(i)}=1\wedge y^{(i)}=0\})}{\sum_{j=1}^m 1\{y^{(i)}=0\}}\\
Therefore, \\
\theta_{j|y=1} &=\frac{\sum_{j=1}^m (1\{x_j^{(i)}=1\wedge y^{(i)}=1\})}{\sum_{j=1}^m 1\{y^{(i)}=1\}}\\
\nabla_{\theta_{y}}L&=\nabla_{\theta_{y}}\sum_{j=1}^m(y^{(i)}log\theta_y+(1-y^{(i)}log(1-\theta_y))\\
0&=\sum_{j=1}^m(y^{(i)}(1-\theta_y)+(1-y^{(i)}\theta_y)\\
&=\sum_{j=1}^m y^{(i)}-\sum_{j=1}^m \theta_y\\
\theta_y&=\frac{\sum_{j=1}^m 1\{y^{(i)}=1\}}{m}
\end{align*}


\end{enumerate}

\end{document}
